\documentclass[a4paper]{article}
\usepackage{amsmath, amssymb, amsthm, mathrsfs}
\usepackage[hmargin = 1.25in, vmargin = 1in]{geometry}
\title{Homework-3}
\author{Haocheng Wang \and 2019011994}

\newtheorem{ex}{Exercise}[subsection]
\stepcounter{section}
\stepcounter{subsection}
\renewcommand{\proofname}{\noindent\bf Proof}

\begin{document}
\maketitle
\setcounter{subsection}{2}
\setcounter{ex}{6}

\begin{ex}[Crieria for measurability]\end{ex}
\begin{proof}
(i) $\iff$ (ii) is the defintion of Lebesgue measure.\\
(ii) $\Rightarrow$ (iii) is trivial because one can always set $U \supset E$ thus $U \Delta E = U \setminus E$.
(iii) $\Rightarrow$ (ii) For every $\varepsilon > 0$, there exists an open set $V$ such that 
$m^*(V \Delta E) \leq \varepsilon/3$. In particular $m^*(E \setminus V) \leq m^*(E \Delta V) \leq 
\varepsilon/2 < 2\varepsilon /3$. Then one can find another open set $W$ containing $E \setminus V$ such that
$m^*(W) \leq 2\varepsilon / 3$. Let $U = V \cup W$, hence\begin{align*}
m^*(U \setminus E) = m^*((V \setminus E) \cup (W \setminus E)) \leq m^*(V \setminus E) + m^*(W \setminus E)
\leq \frac{\varepsilon}{3} + \frac{2\varepsilon}{3} = \varepsilon.
\end{align*}
(i) $\Rightarrow$ (iv) For every $\varepsilon > 0$, since $E$ is measurable, $E^c$ is also measurable, therefore
there exists an open set $U \supset E^c$ such that $m^*(U \setminus E^c) \leq \varepsilon$. Let $F = U^c$ is a 
closed set, then$$
m^*(E \setminus F) = m^*(U \cap E) = m^*(U \setminus E^c) \leq \varepsilon.
$$
(iv) $\Rightarrow$ (v) is trivial since one can set closed set $F \subset E$.\\
(v) $\Rightarrow$ (vi) holds since all of closed sets art measurable.\\
(vi) $\Rightarrow$ (iii) For every $\varepsilon > 0$, there exists measurable set $D$ such that 
$m^*(D \Delta E) \leq \frac{\varepsilon}{2}$. Then one can find an open set $U$ containing $D$ such that
$m^*(U \setminus D) \leq \frac{\varepsilon}{2}$. Note that\begin{align*}
U\Delta E &= (U \setminus E) \cup (E \setminus U) = (D\setminus E) \cup ((U \setminus D) \setminus E) \cup (E \setminus U)\\
&\subset (D \setminus E) \cup (U \setminus D) \cup (E \setminus D) = (D\Delta E) \cup (U \setminus D),
\end{align*}
and hence by subadditivity and monotonicity one has $$
m^*(U \Delta E) \leq m^*(D \Delta E) + m^*(U \setminus D) \leq \varepsilon.
$$
\end{proof}

\begin{ex}\end{ex}
\begin{proof}
Given that $E$ is Jordan measurable set, there exists elementary sets $A \subset E \subset B$ such that 
$m^*(B \setminus A) \leq \frac{\varepsilon}{3}$. Since $A, B$ are elementary, one can always find an open set $B' \supset B$
and a closed set $A' \subset A$ such that $m^*(B' \setminus A') \leq \varepsilon$. Hence $$
B' \setminus E \subset B' \setminus A' \Rightarrow m^*(B' \setminus E) \leq m^*(B' \setminus A') \leq \varepsilon.
$$That is $E$ is Lebesgue measurable.
\end{proof}

\setcounter{ex}{10}
\begin{ex}[Monotone convergence theorem for measurable sets]\end{ex}
\begin{proof}
(i)(Upward monotone convergence) Express $\bigcup_{n = 1}^\infty E_n$ as the countable union of disjoint sets 
$\bigcup_{n = 1}^\infty (E_n \setminus E_{n - 1})$, by countable additivity one has that $$
m\left( \bigcup_{n = 1}^\infty E_n \right) = m\left( \bigcup_{n = 1}^\infty (E_n \setminus E_{n - 1}) \right)
= \sum_{n = 1}^\infty m(E_n \setminus E_{n - 1}) = \lim_{N \to \infty} \sum_{n = 1}^N m(E_n \setminus E_{n - 1}).
$$Note that $\bigcup_{n = 1}^N (E_n \setminus E_{n - 1}) = E_N$, thus $$
m\left( \bigcup_{n = 1}^\infty E_n \right) = \lim_{N \to \infty} \sum_{n = 1}^N m(E_n \setminus E_{n - 1})
= \lim_{n \to \infty} m(E_n).
$$
(ii)(Downward monotone convergence) We can write each $E_n$ as a disjoint union $$
E_n = \bigcup_{k = n}^\infty (E_k \setminus E_{k + 1}) \cup \bigcap_{k = 1}^\infty E_k.
$$Then by additivity,$$
m(E_n) = \sum_{k = n}^\infty m(E_k \setminus E_{k + 1}) + m\Big(\bigcap_{k = 1}^\infty E_k \Big)
$$The statement that at least one of the $m(E_n)$ is finite ensures that series $\sum_{k = n}^\infty m(E_k \setminus E_{k + 1})$ converges
and $m(\bigcap_{k = 1}^\infty B_k)$ is finite. Thus,$$
\lim_{n \to \infty} m(E_n) = \lim_{n \to \infty} \sum_{k = n}^\infty m(E_k \setminus E_{k + 1}) + 
m\Big(\bigcap_{k = 1}^\infty E_k \Big) = m\Big(\bigcap_{k = 1}^\infty E_k \Big).
$$
(iii) Let $A_1 = [0, \frac{1}{2}), A_2 = [\frac{1}{2}, \frac{3}{4}), \dots, A_n = [1 - 2^{1 - n}, 1 - 2^{-n}), \dots$,
then we take the union of their translation: $E_n = \bigcup_{k = 0}^\infty (A_n + k)$. It's obvious that 
$\bigcap_{n = 1}^\infty E_n = \emptyset, m(E_n) = +\infty$. Therefore$$
m\Big(\bigcap_{n = 1}^\infty E_n\Big) = 0 \ne +\infty = \lim_{n \to \infty} m(E_n).
$$
\end{proof}

\setcounter{ex}{14}
\begin{ex}[Inner regularity]\end{ex}
\begin{proof}
Firstly suppose that $E$ is bounded, by the conclusion in {\bfseries Exercise 1.2.7} for every $\varepsilon > 0$
there exists a closed set $K \subset E$ such that $m^*(E \setminus K) \leq \varepsilon$. Since $K$ is closed and
bounded, $K$ is compact, moreover$$
m(K) = m(E) - m(E \setminus K) \geq m(E) - \varepsilon
$$holds for arbitary $\varepsilon > 0$. Hence $m(E) \leq \sup_{K \subset E} m(K)$. The reverse direction of the 
inequality is obvious by the monotonicity.

Now we assume that $E$ is not bounded, let $E_n = E \cap (B(0, n) \setminus B(0, n - 1))$, then for every 
$\varepsilon > 0$ there exists a compact set $K_n \subset E_n$ such that $m(E_n \setminus K_n) \leq \varepsilon 2^{-n}$.
Thus $$
m\Big(\bigcup_{n = 1}^N K_n\Big) = \sum_{n = 1}^N m(K_n) \geq \sum_{n = 1}^N m(E_n) - \varepsilon \to m(E) - \varepsilon.
$$
This implies that there exists $N$ large enough such that $m(\bigcup_{n = 1}^N K_n) \geq m(A) - 2\varepsilon.$
Hence $m(A) \leq \sup_{K \subset E} m(K)$. The proof has been completed.
\end{proof}

\begin{ex}[Crieria for finite measure]\end{ex}
\begin{proof}
(i) $\Rightarrow$ (ii) Since $E$ is Lebesgue measurable, there exists an open set $U$ containing $E$ such that
$m^*(U \setminus E) \leq  \varepsilon$. Note that $U \setminus E$ and $E$ are disjoint, thus 
$m(U) = m(U \setminus E) + m(E)$ is finite.

(ii) $\Rightarrow$ (iii) For every $\varepsilon > 0$, there exists an open set $U \sup E$ suchu that 
$m^*(U \setminus E) \leq \varepsilon / 2$. Let $U_n = U \cap (B(0, n) \setminus B(0, n - 1))$, then 
$E \Delta \bigcup_{k = 1}^n U_k = (U \setminus E) \cup \bigcup_{k = n + 1}^\infty U_k$. Set $n$ large enough such that 
$m^*(\bigcup_{k = n + 1}^\infty U_n) \leq \varepsilon / 2$.  Thus $\bigcup_{k = 1}^n U_k$ is an open set and$$
m^*\Big(E \Delta \bigcup_{k = 1}^\infty U_k \Big) \leq m^*(U \setminus E) + m^*\Big(\bigcup_{k = n + 1}^\infty U_n\Big) 
\leq \varepsilon.
$$

(iii) $\Rightarrow$ (i) For every $\varepsilon > 0$, there exists a bounded open set $V$ such that 
$m^*(E \Delta V) \leq \varepsilon / 2$, then there exists an open set $W \supset E \Delta V$ such that 
$m^*(W) \leq \varepsilon$. Let $U = V \cup W$, then this open set contains $E$, and $$
m^*(U \setminus E) \leq m^*(W) \leq \varepsilon.
$$Hence $E$ is Lebesgue measurable, furthermore, by monotonicity, 
$E \subset V \cup W \Rightarrow m(E) \leq m(V) + m(W) < +\infty$.

(i) $\Rightarrow$ (iv) By {\bfseries Exercise 1.2.7}(iv), for every $\varepsilon > 0$ there exists a closed set
$F \subset E$ such that $m^*(E \setminus F) \leq \varepsilon /2$. Let $F_n = F \cap \overline{B(0, n) \setminus B(0, n - 1)}$.
Then $E$ can be expressed as the union of disjoint sets 
$E = (E \setminus F) \cup (F \setminus \bigcup_{n = 1}^N F_n) \cup \bigcup_{n = 1}^N F_n$. By upward monotone convergence,
one can always choose $N$ large enough such that $m(F \setminus \bigcup_{n = 1}^\infty F_n) \leq \varepsilon / 2$.
Thus $\bigcup_{n = 1}^\infty F_n \subset E$ and satisfies that $$
m\Big(E \setminus \bigcup_{n = 1}^\infty F_n\Big) = m^*(E \setminus F) + m\Big(F \setminus \bigcup_{n = 1}^\infty F_n\Big)
\leq \varepsilon.
$$

(iv) $\Rightarrow$ (v) This is trivial.

(v) $\Rightarrow$ (vii) This is trivial for any compact sets are Lebesgue measurable sets with finite measure.

(vii) $\Rightarrow$ (vi) This can be easily proved by using the method in showing that (i) $\Rightarrow$ (iv).

(vi) $\Rightarrow$ (viii) By assumption, for every $\varepsilon > 0$, there exists a bounded measurable set $D$
such that $m^*(E \Delta D) \leq \varepsilon / 2$. Then by the defintion of $m^*$, there exists an elementary set $A$
such that $m^*(D \Delta A) \leq \varepsilon / 2$. In fact, there exists a sequence of almost disjoint boxes $\{B_n\}$ 
such that $\bigcup_{n = 1}^\infty B_n \supset D, m^*(\bigcup_{n = 1}^\infty B_n) - m^*(D) \leq \varepsilon / 4$. Since
$D$ is Lebesgue measurable, then by countable additivity $m^*(\bigcup_{n = 1}^\infty B_n \setminus D) \leq \varepsilon / 4$.
By assumption $D$ is bounded, so $\sum_{n = 1}^\infty |B_n| < +\infty$, then there exists $N \in \mathbb{N}$ such
that $\sum_{n = N + 1}^\infty |B_n| \leq \varepsilon / 4$. Let $A = \bigcup_{n = 1}^N B_n$, note that$$
A \Delta D = (A \setminus D) \cup (D \setminus A) \subset \Big(\bigcup_{n = 1}^\infty B_n \setminus D\Big) \cup
\Big(\bigcup_{n = N + 1}^\infty B_n\Big),
$$thus $m^*(A \Delta D) \leq \varepsilon / 2$, as we have asserted. Now \begin{align*}
A \Delta E &= (A \setminus E) \cup (E \setminus A) \subset [(A \setminus D) \setminus E] \cup (D \setminus E) \cup 
[(E \setminus D) \setminus A] \cup (D \setminus A)\\
&\subset (A \setminus D) \cup (D \setminus E) \cup (E \setminus D) \cup (D \setminus A) = (A \Delta D) \cup (D \Delta E),
\end{align*}
Finally we obtain that$
m^*(A \Delta E) \leq m^*(A \Delta D) + m^*(D \Delta E) \leq \varepsilon.
$

(viii) $\Rightarrow$ (ix) Given an elementary set $A$ such that $m^*(A \Delta E) \leq \varepsilon / 2$, it suffices
to show that there exists $n \in \mathbb{N}$ a finite union $F$ of closed dyadic cubes of sidelength $2^{-n}$ such
that $m^*(F \Delta A) \leq \varepsilon / 2$, and this can be ensured by {\bfseries Exercise 1.1.14}. Then$$
m^*(E \Delta F) \leq m^*(E \Delta A) + m^*(A \Delta F) \leq \varepsilon,
$$for $\varepsilon$ is arbitarily small, we complete the proof.

(ix) $\Rightarrow$ (iii) For every $\varepsilon > 0$, there exists an integer $n$ and a finite union $F$ of closed 
dyadic cubes of sidelength $2^{-n}$ such that $m^*(E \Delta F) \leq \varepsilon / 2$. Express $F$ as
$F = \bigcup_{i = 1}^m F_i$, now we enlarge each of $F_i$ to open set $F_i'$ such that 
$m^*(F_i' \setminus F_i) \leq \frac{\varepsilon}{2m}$. Then $F' = \bigcup_{i = 1}^m F_i'$ is a bounded open set satisfying
$m^*(F' \Delta F) \leq \varepsilon / 2$. Hence\[
m^*(E \Delta F') \leq m^*(E \Delta F) + m^*(F \Delta F') \leq \varepsilon.\qedhere
\]
\end{proof}

\begin{ex}[Carath\'eodory criterion, one direction]\end{ex}
\begin{proof}
(i) $\Rightarrow$ (iii) Since both $B \cap E$ and $B \setminus E$ are Lebesgue measurable, then by countable additivity
one has that $|B| = m(B \cap E) + m(B \setminus E)$.

(iii) $\Rightarrow$ (ii) Express $A = \bigcup_{n = 1}^N B_n$ for disjoint boxes $\{B_n\}_{n = 1}^N$, then by subadditivity
\begin{align*}
m^*(A \cap E) + m^*(A \setminus E) &= m^*\Big( \bigcup_{n = 1}^N (B_n \cap E)\Big) + m^*\Big(\bigcup_{n = 1}^N 
(B_n \setminus E)\Big) \\
&\leq \sum_{n = 1}^N m^*(B_n \cap E) + \sum_{n = 1}^N m^*(B_n \setminus E) = \sum_{n = 1}^N |B_n| = m(A).
\end{align*}
The inequality $m(A) \leq m^*(A \cap E) + m^*(A \setminus E)$ is obvious by subadditivity. Hence 
$m(A) = m^*(A \cap E) + m^*(A \setminus E)$.

(ii) $\Rightarrow$ (i) For every $\varepsilon > 0$ we let $\{B_n\}_{n \geq 1}$ be disjoint boxes such that 
$\sum_{n = 1}^\infty |B_n| \leq m^*(E) +  \varepsilon$. And then we let open box $B_n' \supset B_n$ such that 
$|B_n'| \leq |B_n| + \varepsilon 2^{-n}$, and set $U = \bigcup_{n = 1}^\infty B_n'$ be an open set. Thus by 
subadditivity, $$
m^*(E) + m^*(U \setminus E) \leq \sum_{n = 1}^\infty [m^*(B_n' \cap E) + m^*(B_n' \setminus E)] 
\leq \sum_{n = 1}^\infty |B_n| + \varepsilon = m^*(E) + \varepsilon.
$$That is $m^*(U \setminus E) \leq \varepsilon$, $E$ is Jordan measurable.
\end{proof}

\stepcounter{ex}
\begin{ex}\end{ex}
\begin{proof}
(i) $\Rightarrow$ (ii) Since $E$ is Lebesgue measurable, then one can find 
an open set $U_n \supset E$ such that $m^*(U_n \setminus E) \leq \frac{1}{n}$. Let $U = \bigcap_{n = 1}^\infty U_n$, then 
$U \setminus E \subset U_n \setminus E, \forall n \geq 1$, thus $m^*(U \setminus E) \leq m^*(U_n \setminus E) \leq \frac{1}{n}$,
let $n \to \infty$, we obtain that $m^*(U \setminus E) = 0$. Note that $E = U\setminus (U \setminus E)$ where $U$ is $G_\delta$
and $U \setminus E$ is a null set we finish the proof.

(ii) $\Rightarrow$ (i) Assume that $E = G \setminus N$ where $G = \bigcap_{n = 1}^\infty G_n$ is a $G_\delta$ set 
and $N$ is a null set. Set $U_n = \bigcap_{k = 1}^n G_k$, them $\{U_n\}_{n \geq 1}$ is a non-increasing sequence.
Note that$$
U_n \setminus E = U_n \cap (G \cap N^c)^c = (U_n \cap G^c) \cup (U_n \cap N) = (U_n \setminus G) \cup (U_n \cap N),
$$since $G \subset U_n$ is the countable intersection of Lebesgue measurable set and $U_n \cap N \subset N$, 
$G$ is Lebesgue measurable and $U_n \cap N$ is also a null set. Now we can conclude that $$
m^*(U_n \setminus E) \leq m^*(U_n \setminus G) + m^*(U_n \cap N) = m^*(U_n) - m^*(G) = m^*(U_n) - 
m^*\Big(\bigcap_{n = 1}^\infty U_n\Big) \to 0,
$$here we have use the conclusion $m^*(\bigcap_{n = 1}^\infty U_n) = \lim_{n \to \infty} m^*(U_n)$.  Hence $E$ is
Lebesgue measurable.

(i) $\Rightarrow$ (iii) Since $E$ is Lebesgue measurable, its complement in $\mathbb{R}^d$ $E^c$ is also Lebesgue
measurable, thus we can express $E^c$ as $E^c = G \setminus N$ where $G$ is a $G_\delta$ set and $N$ is a null set.
Therefore $E = (G \cap N^c)^c = G^c \cup N$, where the complement $G^c$ is a $F_\sigma$ set.

(iii) $\Rightarrow$ (i) Suppose that $E = F \cup N$ where $F$ is a $F_\sigma$ set and $N$ is a null set, then 
$E^c = F^c \setminus N$ where $F^c$ is a $G_\delta$ set, by (ii) $E^c$ is Lebesgue measurable, thus $E$ is also 
Lebesgue measurable.
\end{proof}

\begin{ex}[Translation invariance]\end{ex}
\begin{proof}
First we show that Lebesgue outer measure is translation invariance. Let $\{B_n\}_{n \geq 1}$ be a sequence of boxes
covering $E$, then $\{B_n + x\}_{n \geq 1}$ cover $E + x$ for any $x \in \mathbb{R}^d$. Therefore $$
m^*(E + x) \leq \sum_{n = 1}^\infty |B_n + x| = \sum_{n = 1}^\infty |B_n|.
$$
So for every sequence of boxes $\{B_n\}_{n \geq 1}$ that cover $E$ one has $m^*(E + x) \leq \sum_{n = 1}^\infty |B_n|$,
hence $m^*(E + x) \leq m^*(E)$. Similarly we can prove the reverse inequality by take the reverse translation 
$E = (E + x) - x$, and we obtain that $m^*(E) = m^*(E + x)$  holds for any $E \subset \mathbb{R}^d$ and $x \in \mathbb{R}^d$.

Suppose that $E$ is Lebesgue measurable, then for arbitary $\varepsilon > 0$ there exists an open set $U \supset E$ 
such that $m^*(U \setminus E) \leq \varepsilon$. Note that 
$m^*((U + x) \setminus (E + x)) = m^*(U \setminus E + x) = m^*(U \setminus E) \leq \varepsilon$, so $E + x$ is 
also Lebesgue measurable, and that $m(E + x) = m(E)$.
\end{proof}

\setcounter{ex}{22}
\begin{ex}[Uniqueness of Lebesgue measure]\end{ex}
\begin{proof}
From non-negativity and countable additivity we conclude the monotonicity property, and from monotonicity and 
additivity we obtain subadditivity. And as we have proved, $m$ must match elementary measure on elementary set.

{\bfseries Step I.} First we show that for any null set $E$, $m(E) = 0$. Since $E$ is measure zero, for every 
$\varepsilon > 0$ there exists a sequence of boxes $\{B_n\}_{n \geq 1}$ such that 
$E \subset \bigcup_{n = 1}^\infty B_n, m^*(\bigcup_{n = 1}^\infty B_n) \leq \varepsilon$, note that $m^*(B_n) = m(B_n)$:
$$
m(E) \leq m\Big(\bigcup_{n = 1}^\infty B_n\Big) = \sum_{n = 1}^\infty m(B_n) = \sum_{n = 1}^\infty m^*(B_n) \leq \varepsilon.
$$
Let $\varepsilon$ tends to zero and we obtain that $m(E) = 0$.

{\bfseries Step II.} Now we show that for every open set $U$ of finite measure, then $m(U) = m^*(U)$. Express 
$U$ as the union of almost disjoint closed boxes $U = \bigcup_{n = 1}^\infty B_n$, then $B_n^\circ$ are pairwisely
disjoint. Let $F = \bigcup_{n = 1}^\infty B_n \setminus \bigcup_{n = 1}^\infty B_n^\circ$. Since $$
m^*(F) = m^*(U) - m^*\Big(\bigcup_{n = 1}^\infty B_n^\circ\Big) = \sum_{n = 1}^\infty |B_n| - \sum_{n = 1}^\infty
|B_n^\circ| = 0,
$$one has that $m(F) = 0$. Note that $U$ is the union of disjoint sets $F$ and $\bigcup_{n = 1}^\infty B_n^\circ$, thus
$$
m(U) = m(F) + m\Big(\bigcup_{n = 1}^\infty B_n^\circ\Big) = m(F) + \sum_{n = 1}^\infty |B_n^\circ| = \sum_{n = 1}^\infty
|B_n^\circ| = m^*(U).
$$

{\bfseries Step III.} We show that $m$ is bounded by Lebesgue outer measure $m^*$. Given Lebesgue measurable set
$E$ with $m^*(E) < \infty$, for every $\varepsilon > 0$ there exists an open set $U \supset E$ such that
$m^*(U \setminus E) \leq \varepsilon \Rightarrow m^*(U) - m^*(E) \leq \varepsilon$. Then by monotonicity 
$m(E) \leq m(U) = m^*(U) \leq m^*(E) + \varepsilon$. Let $\varepsilon \to 0$ we obtain that $m(E) \leq m^*(E)$.

{\bfseries Step IV.} We show that for any Lebesgue measurble bounded set $E$ we have $m(E) = m^*(E)$. Let $B$ be
a box such that $E \subset B$. By additivity we have $m(B \setminus E) = (B) - m(E)$. Also$$
m(E) \leq m^*(E) = m^*(B) - m^*(B \setminus E) \leq m^*(B) - m(B \setminus E) = m(B) - (m(B) - m(E)) = m(E).
$$Hence $m(E) = m^*(E)$.

{\bfseries Step V.} Finally we show that for any Lebesgue measurable set $E$ one has that $m(E) = m^*(E)$. Decompose
$E$ as $E = \bigcup_{n = 1}^\infty E_n$, where $E_n = E \cap (\overline{B(0, n)} \setminus \overline{B(0, n - 1)})$
is a bounded set.
Then by countable additivity of both $m$ and $m^*$, \[
m(E) = \sum_{n = 1}^\infty m(E_n) = \sum_{n = 1}^\infty m^*(E_n) = m^*(E).\qedhere
\]
\end{proof}

\begin{ex}[Lebesgue measure as the completion of elementary measure]\end{ex}
\begin{proof}
(i) We verify the relation by the defintion of equivalence relation:\begin{itemize}
    \item (Reflexivity) This is obvious for $E \Delta E = \emptyset, \forall E \in 2^A$.
    \item (Symmetry) Note that $E \Delta F = (E \setminus F) \cup (F \setminus E) = (F \setminus E) \cup(E \setminus F) = F \Delta E$.
    \item (Transitivity) Assume that $A, B, C \in 2^A$ satisfying $m^*(A \Delta B) = m^*(B \Delta C) = 0$. Note
    that $A \Delta C \subset (A \Delta B) \cup (B \Delta C) \Rightarrow  m^*(A \Delta C) \leq m^*(A \Delta B)
    + m^*(B \Delta C) = 0 \Rightarrow m^*(A \Delta C) = 0$.
\end{itemize}
Hence this is a equivalence relation.

(ii) From the defintion of the relation $\sim$, both $(E \setminus F) \cup (F \setminus E)$ and 
$(E' \setminus F') \cup (F' \setminus E')$ are null set, therefore $E \setminus F, E' \setminus F'$ are null
sets. Thus one has the expressions $F = E_1 \cup N, F' = E_1' \cup N'$ where $E_1 = F \cap E, E_1' = F' \cap E'$ and
$N, N'$ are null sets. Now $$
F \Delta F' = (E_1 \cup N) \Delta (E_1' \cup N') = \big((E_1 \cup N) \cap \big(E_1' \cup N')^c\big) \cup 
\big((E_1' \cup N') \cap (E_1 \cap N)^c\big).
$$Consider the first term$$
(E_1 \cup N) \cap (E_1' \cup N')^c = (E_1 \cup N) \cap (E_1'^c \cap N'^c) \subset (E_1 \cup N) \cap E_1'^c
= (E_1 \cap E_1'^c) \cup (E_1'^c \cap N),
$$we can conclude similar result for the second term, hence $$
F \Delta F' \subset (E_1 \Delta E_1') \cup (E_1'^c \cap N) \cup (E_1^c \cap N') \subset (E \Delta E') \cup N \cup N'.
$$The right hand side is the union of $E \Delta E'$ and two null sets, thus $m^*(F \Delta F) \leq m^*(E \Delta E')$.
Symmetrically we can proved the reverse inequality, then $m^*(E \Delta E') = m^*(F \Delta F')$ whenever $[E] = [F]$
and $[E'] = [F']$, hence the distance is well defined.

(iii) Given a sequence $\{[E_n]\} \in \mathcal{E} / \sim$ and $[E] \in \mathcal{E} /\sim$ being a limit of $\{[E_n]\}$,
i.e., for every $\varepsilon > 0$, there exists $N \in \mathbb{N}$ such that $d([E], [E_n]) \leq \varepsilon, \forall n > N$.
Assert that $[E] \in \mathcal{L} / \sim$. It suffices to show that the representative $E$ is Lebesgue measurable.
By assumption, $d([E], [E_n]) \leq \varepsilon$, that is, $m^*(E \Delta E_n) \leq \varepsilon \Rightarrow
m^*(E \setminus E_n) \leq \varepsilon$. Thus there exists an open set $V \supset E \setminus E_n$ such that 
$m^*(V) \leq 2\varepsilon$. Since $m^*(E_n \Delta E_n^\circ) = 0$, one can always choose representative $E_n$ as
an open set. Let $U = V \cup E_n$ be an open set, then $$
m^*(U \setminus E) \leq m^*(V \setminus E) + m^*(E_n \setminus E) \leq m^*(V) + m^*(E_n \Delta E) \leq 3\varepsilon.
$$Hence $E$ is Lebesgue measurable, i.e., $[E] \in \mathcal{L} / \sim$.

(iv) Given $E, F \in \mathcal{L}$, note that$$
|m(E) - m(F)| = |m(E) - m(F \cap E) - m(F \setminus E)| = |m(E \setminus F) - m(F \setminus E)| \leq m(E \Delta F),
$$so if $E \sim  F$, $m(E) = m(F)$, i.e., $m: \mathcal{L} / \sim \to \mathbb{R}^+$ is well defined. Moreover,$$
|m([E_1]) - m([E_2])| \leq d([E_1], [E_2]).
$$Thus, $m$ is a continuous funcion on $\mathcal{L}/ \sim$, in particular, it's uniformly continuous.
\end{proof}
\end{document}