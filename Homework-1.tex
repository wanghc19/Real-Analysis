\documentclass{article}
\usepackage{amsmath, amssymb, amsthm}
\usepackage[hmargin = 1.25in, vmargin = 1in]{geometry}
\author{Haocheng Wang \and 2019011994}
\title{Homework-1}
%\everymath{\displaystyle}

\begin{document}
\maketitle
\newtheorem{ex}{Exercise}[subsection]
\stepcounter{section}
\stepcounter{subsection}
\begin{ex}[Boolean Closure]\end{ex}
\begin{proof}
Since both of $E$ and  $F$ is elementary sets which are the union of finite number of boxes, we express $E$ and $F$
as unions of boxes $$
E = \bigcup_{i = 1}^m E_i,\ \ \ F = \bigcup_{j = 1}^n F_j.
$$
Note that the union of $E$ and $F$ is $\displaystyle \Big(\bigcup_{i = 1}^m E_i\Big) \cup \Big(\bigcup_{j = 1}^n F_j\Big)$,
which is also the union of finite boxes. Hence the union $E \cup F$ is an elementary set. Similarly,$$
E \cap F = \Big(\bigcup_{i = 1}^m E_i\Big) \cap \Big(\bigcup_{j = 1}^n F_j\Big) = \bigcup_{i = 1}^m \bigg( E_i 
\cap \Big( \bigcup_{j = 1}^n F_j\Big) \bigg) = \bigcup_{i = 1}^m \bigcup_{j = 1}^n (E_i \cap F_j),
$$Observe that the intersection of finite boxes is still a box, then the intersection of $E$ and $F$ is elementary set.
For the similar reason in the case of intersection, it suffices to show that the set theoretic difference of any two
boxes $E_i$ and $F_j$ is still elementary set. In fact $E_i \setminus F_j = E_i - (E_i \cap F_j)$, which can be 
expressed as the union of some boxes. To prove $E \Delta F$ is elementary set, we note that both $E \setminus F$
and $F \setminus E$ are elementary sets, then their union is still elementary set. It's obviously that the translation 
of a box is still a box, then the translation $E + x$ is also elementary set.
\end{proof}

\stepcounter{ex}
\begin{ex}[Uniqueness of elementary measure]\end{ex}
\begin{proof}
Since any elementary set $E$ can be expressed as the finite union of disjoint boxes, it suffices to show that 
$m'(B) = cm(B)$ for any boxes $B \subset \mathbb{R}^d$, where $c := m'([0, 1)^d)$. From the finite additivity and 
translation invariance properties we obtain that $m'([0, \frac{1}{n})) = \frac{c}{n^d}$, then for all $r \in \mathbb{Q}$ 
we have $m([0, r)^d) = cr^d$. Now we consider the irrational number $i \in \mathbb{R} \setminus \mathbb{Q}$, then 
we can find two sequences of rational number $\{s_n\}, \{t_n\}$ that $\{s_n\}$ is monotonically increasingly tends to $i$ 
and $\{t_n\}$ is monotonically decreasingly tends to $i$, then from the non-negativity property we have $$
cs_n^d = m([0, s_n)^d) \leq m([0, i)^d) \leq m([0, t_n)^d) = ct_n^d.
$$Let $n \to \infty$, we get $m([0, i)^d) = ci^d, i \in \mathbb{R} \setminus \mathbb{Q}$. Since we can build any 
boxes by using cube $[0, x)^d, x \in \mathbb{R}$, we finally obtain $m'(E) = cm(E)$ for any elementary sets $E$.
\end{proof}

\stepcounter{ex}
\begin{ex}[Charaterisation of Jordan measurablity]\end{ex}
\begin{proof}
$(1) \Rightarrow (2)$ Since $E$ is Jordan measurable, supremum $m_{*,(J)}(E)$ and infimum $m^{*,(J)}(E)$ exist. 
That is, there exists elementary sets $A$ and $B$ satisfying $A \subset E \subset B$ s.t. for every 
$\varepsilon > 0$$$
m_{*, (J)}(E) - m(A) \leq \frac{\varepsilon}{2},\ \ m(B) - m^{*, (J)}(E) \leq \frac{\varepsilon}{2} \Rightarrow 
m(B \setminus A) \leq \varepsilon.
$$
$(2) \Rightarrow (3)$ Note that $A \Delta E = E \setminus A \subset B \setminus A$, from the monotonicity we find
that \[
m^{*,(J)}(A \Delta E) \leq m^{*,(J)}(B \setminus A) = m(B \setminus A) \leq \varepsilon.
\]
$(3) \Rightarrow (1)$ Since $m^{*,(J)}(A \Delta E) \leq \varepsilon$, there exists a elementary set 
$C \supset A \Delta E$ s.t. $m(C) \leq 2\varepsilon$, then we find that $A \setminus C \subset E \subset A \cup C$
where $m((A \cup C) - (A \setminus C)) \leq 4\varepsilon$. Notice that $$
m(A \setminus C) \leq m_{*,(J)}(E) \leq m^{*, (J)}(E) \leq m(A \cup C) \leq m(A \setminus C) + 4\varepsilon,
$$let $\varepsilon$ tends to 0, we obtain $m_{*,(J)}(E) = m^{*, (J)}(E)$. That is, $E$ is Jordan measurable.
\end{proof}

\begin{ex}\end{ex}
\begin{proof} 
(1)(Boolean Closure) There exists elementary sets $A, B, C, D$ s.t. for every $\varepsilon > 0$$$
A \subset E \subset B,\ C \subset F \subset D,\ m(B \setminus A) \leq \varepsilon/2, 
m(D \setminus C) \leq \varepsilon / 2.
$$To prove that $E \cup F$ is Jordan measurable, we note that $A \cup C \subset E \cup F \subset B \cup D$,\begin{align*}
m((B \cup D) \setminus (A\cup C)) &= m((B - A \cup C) \cup (D - A \cup C)) \leq 
m(B - A \cup C) + m(D - A \cup C)\\
&\leq m(B \setminus A) + m(D \setminus C) \leq \varepsilon.
\end{align*}
Hence $E \cup F$ is Jordan measurable. To prove case of $E \cap F$, we notice that $$
(B \cap D) \setminus (A \cap C) = ((B \cap D) \setminus A) \cup ((B \cap D) \setminus C) \subset 
(B \setminus A) \cup (D \setminus C),
$$then$$
m((B \cap D) \setminus (A \cap C)) \leq m(B \setminus A) + m(D \setminus C) \leq \varepsilon.
$$Hence $E \cap F$ is Jordan measurable. To prove the third case, notice that $A \setminus D \subset E \setminus F
\subset B \setminus C$, and$$
B \setminus C - A \setminus D \subset (B \setminus A) \cup (D \setminus C) \Rightarrow 
m(B \setminus C - A \setminus D) \leq m(B \setminus A) + m(D \setminus C) \leq \varepsilon.
$$Hence $E \setminus F$ is also Jordan measurable, and for the same reason $E \Delta F$ is Jordan measurable.\\
(2)(Non-negativity) This one is obvious from the definition of Jordan measure, since elementary measure is non-negative.\\
(3)(Finite additivity) Since $E, F$ are disjoint, we can find \emph{almost} disjoint elementary $A_1, B_1, A_2, B_2$ sets 
s.t. $$
A_1 \subset E \subset B_1,\ A_2 \subset F \subset B_2\ \ \text{where}\ \ B_1^{\circ} \cap B_2^{\circ} = \emptyset.
$$Note that $m(B_i) = m(B_1^{\circ}), i = 1, 2$, from the finite additivity of elementary measure we can easily see 
that $m(E \cup F) = m(E) + m(F)$.\\
(4)(Monotonicity) Note that $m(F) = m((F \setminus E) \cup E) = m(F \setminus E) + m(E)$, and from the non-negativity 
we obtain that $m(E) \leq  m(F)$.\\
(5)(Finite subadditivity) From monotonicity and finite additivity properties we obatain $$
m(E \cup F) = m((E \setminus F) \cup F) = m(E \setminus F) + m(F) \leq m(E) + m(F).
$$
(6)(Translation invariance) It's easy to show this for all of elementary sets are translation invariance.
\end{proof}

\begin{ex}[Regions under graphs are Jordan measurable]\end{ex}
\begin{proof}
(1)Since in a compact metric space, continuous function $f(x)$ is uniformly continuous, i.e. for every $\varepsilon > 0$
and $\forall x_0 \in B$, there exists $\delta > 0$ s.t. $|f(x) - f(x_0)| < \varepsilon / |B|$ whenever $|x - x_0| < \delta$.
We take partition $P:= \{B_i\}_{i \in \mathcal{I}}$ of $B$ with a mesh $\lambda(P) < \frac{\delta}{\sqrt{d}}$, here mesh 
$$\lambda(P) := \max_{B_i \in P} \{|I_i| : I_i\ \text{is the longest side of}\ B_i\}.$$
Then we get a collection of open boxes $\{A_i\}_{i \in \mathcal{I}}$ which covers $B$. Hence the measure of graph $G$ $$
0 \leq m_{*,(J)}(G) \leq m^{*, (J)}(G) \leq m\left( \bigcup_{i \in \mathcal{I}} A_i \right) \cdot \frac{\varepsilon}{|B|} = \varepsilon.
$$Hence the graph $G$ is Jordan measurable in $\mathbb{R}^{d + 1}$ with Jordan measure zero.\\
(2)We still use previous notation, but now we choose two distinguished point for each $B_i \in P$
$\xi_i := \inf_{x \in B_i} f(x), \eta_i := \sup_{x \in B_i} f(x)$. Then we get two collection of elementary sets \begin{gather*}
\{\Xi_i\}_{i \in \mathcal{I}} := \{B_i \times [0, \xi_i]: B_i \in P\}, \ \ 
\{H_i\}_{i \in \mathcal{I}} := \{B_i \times [0, \eta_i]: B_i \in P\}.
\end{gather*}
It's obvious that $\bigcup_{i \in \mathcal{I}} \Xi_i \subset \tilde{G} \subset \bigcup_{i \in \mathcal{I}} H_i$, here
$\tilde{G} := \{(x, t) : x \in B; 0 \leq t \leq f(x)\}$, then we obtain$$
m\left( \bigcup_{i \in \mathcal{I}} \Xi_i \right) \leq m_{*,(J)}(\tilde{G}) \leq m^{*,(J)}(\tilde{G})
\leq m\left( \bigcup_{i \in \mathcal{I}} H_i \right) \leq m\left( \bigcup_{i \in \mathcal{I}} \Xi_i \right) + \varepsilon.
$$Let $\varepsilon$ tends to zero, $\tilde{G}$ is Jordan measurable, i.e. $f$ is de facto Riemann integrable.
\end{proof}

\setcounter{ex}{9}
\begin{ex}\end{ex}
\begin{proof}
We firstly prove the closed case, it suffices to show that closed hemisphere is Jordan measurable.
Let $f: \mathbb{R}^{d - 1} \to \mathbb{R}, x \mapsto \sqrt{r^2 - x^2}$ and $B$ is a box which contains the ball 
$x_1^2 + \cdots + x_{d - 1}^2 = r^2$ of $d - 1$ dimension. Then the set 
$\{(x, t): x \in B; 0 \leq t \leq f(x) \} \subset \mathbb{R}^d$ is Jordan measurable, and this set is the hemisphere.
Hence the closed ball in $\mathbb{R}^d$ is Jordan measurable. To prove the open case, we note that for arbitary open ball
at the origin $\overline{B(0, r)} = \partial B(0, r) \cup B(0, r)$, where $$
\partial B(0, r) = \{(x, f(x)): x \in B\}
$$ has a Jordan measure zero. Hence $B(0, r) = \overline{B(0, r)} \setminus \partial B(0, r)$ is Jordan measurable 
and has the same measure with its closure. Since Euclidean balls are Jordan measurable, it can be approximated by
elementary sets, so we can write the $m(B(x, r)) = c_d(r)r^d$, we will show that $c_d(r)$ is a constant depending
only on $d$. Consider the inscribed cube $C_1$ and circumscribed cube $C_2$ of the ball $B(x, r)$, it's easy to 
write that $|C_1| = (\frac{2r}{\sqrt{d}})^d, |C_2| = (2r)^d$. Then we have the inequality $$
\left( \frac{2}{\sqrt{d}} \right)^d r^d \leq m(B(x, r)) \leq 2^d r^d \Rightarrow 
\left( \frac{2}{\sqrt{d}} \right)^d \leq c_d(r) \leq 2^d.
$$Hence $c_d(r) = c_d$ is a constant depending only on dimension $d$.
\end{proof}

\setcounter{ex}{11}
\begin{ex}\end{ex}
\begin{proof}
Suppose that $E$ is a Jordan null set, then for every $\varepsilon > 0$ there exists an elementary set $A$ 
containing $E$ with Jordan measure $\varepsilon$. For any subset $F \subset E \subset A$, then$$
0 \leq m_{*,(J)}(F) \leq m^{*, (J)}(F) \leq m(A) = \varepsilon,
$$let $\varepsilon$ tends to zero, we finally obatain that $F$ is also a Jordan null set.
\end{proof}

\begin{ex}\end{ex}
\begin{proof}
Since $E$ is Jordan measurable, there exists two sequences of elementary sets $\{A_n\}, \{B_n\}$ s.t. $m(E) - m(A_n) < 
\frac{1}{2n}, m(B_n) - m(E) < \frac{1}{2n}$. On the other hand, from $A_n \subset E \subset B_n$ we find that $$
\frac{1}{N^d}\#(A_n \cap \frac{1}{N}\mathbb{Z}^d) \leq \frac{1}{N^d}\#(E \cap \frac{1}{N}\mathbb{Z}^d) \leq 
\frac{1}{N^d}\#(B_n \cap \frac{1}{N}\mathbb{Z}^d).
$$Let $N \to \infty$, we get $$
m(A_n) \leq \varliminf_{N \to \infty} \frac{1}{N^d}\#(E \cap \frac{1}{N}\mathbb{Z}^d) \leq 
\varlimsup_{N \to \infty} \frac{1}{N^d}\#(E \cap \frac{1}{N}\mathbb{Z}^d) \leq m(B_n).
$$Then let $n \to \infty$, we finally obtain that $$
m(E) = \lim_{N \to \infty} \frac{1}{N^d}\#(E \cap \frac{1}{N}\mathbb{Z}^d)
$$holds for all Jordan measurable set $E \subset \mathbb{R}^d$.
\end{proof}

\begin{ex}[Metric entropy fomulation of Jordan measurablity]\end{ex}
\begin{proof}
\emph{Sufficiency}. Let $E_*(E, 2^{-n})$ denotes the collection of dyadic cubes of sidelength $2^{-n}$ that are contained 
in $E$, and let $E^*(E, 2^{-n})$ be the collection of dyadic cubes of sidelength $2^{-n}$ that intersect $E$.
The first observation is that $\{2^{-dn} \mathcal{E}_*(E, 2^{-n})\}$ is non-decreasing, and 
$\{2^{-dn} \mathcal{E}^*(E, 2^{-n})\}$ is non-increasing. Since $
\bigcup_{E_i \in E_*(E, 2^{-n})} E_i \subset E \subset \bigcup_{E_j \in E^*(E, 2^{-n})} E_j, 
$ it's obvious that $2^{-dn} \mathcal{E}_*(E, 2^{-n}) \leq 2^{-dn} \mathcal{E}^*(E, 2^{-n})$, hence both of 
$\lim_{n \to \infty} 2^{-dn} \mathcal{E}_*(E, 2^{-n})$ and $\lim_{n \to \infty} 2^{-dn} \mathcal{E}^*(E, 2^{-n})$
exist. Since $\lim_{n \to \infty} (2^{-dn} \mathcal{E}^*(E, 2^{-n}) - 2^{-dn} \mathcal{E}_*(E, 2^{-n})) = 0$, for
every $\varepsilon > 0$, there exists $N \in \mathbb{N}$ s.t. $$
m\Bigg( \bigcup_{E_j \in E^*(E, 2^{-n})} E_j \setminus \bigcup_{E_i \in E_*(E, 2^{-n})} E_i \Bigg) \leq \varepsilon,
\forall n > N.
$$
Hence $E$ is Jordan measurable, in which case one has $$
m(E) = \lim_{n \to \infty} 2^{-dn} \mathcal{E}_*(E, 2^{-n}) = \lim_{n \to \infty} 2^{-dn} \mathcal{E}^*(E, 2^{-n}).
$$
\emph{Necessity}. To solve this problem, we firstly define the enlargement and shrinking of any given boxes 
$B = [a_1, b_1] \times [a_2, b_2] \times \cdots \times [a_d, b_d]$ at scale $2^{-n}$\begin{itemize}
    \item The enlargement of $B$ at scale $2^{-n}$ is $(B)^n := [a_1 - 2^{-n}, b_1 + 2^{-n}] \times \cdots \times 
    [a_d - 2^{-n}, b_d + 2^{-n}]$;
    \item The shrinking of $B$ at scale $2^{-n}$ is $(B)_n := [a_1 + 2^{-n}, b_1 - 2^{-n}] \times \cdots \times 
    [a_d + 2^{-n}, b_d - 2^{-n}]$.
\end{itemize}
Since $E$ is Jordan measurable, one can find two elementary set $A, B$ s.t. $A \subset E \subset B, m(B \setminus A) < \varepsilon$,
then rewrite $A$ and $B$ as the finite unions of disjoint boxes $A = \bigcup_i A_i, B = \bigcup_j B_j$. We are now
going to consider an arbitary dyadic cube $D$ which is contained in $\bigcup_{E_j \in E^*(E, 2^{-n})} E_j \setminus 
\bigcup_{E_i \in E_*(E, 2^{-n})} E_i$. It's easy to find that $D \cap (A_i)_n = \emptyset$ for all $i$, and $D$ is
contained in some $(B)^n$s for $D$ is intersecting with $B$. Thus, $D \in \bigcup_j (B_j)^n - \bigcup_i (A_i)_n$.
And we find that for any Jordan measurable set $F$ the inequalities $$
2^{-dn} \mathcal{E}_*(F, 2^{-n}) \leq m(F) \leq 2^{-dn} \mathcal{E}^*(F, 2^{-n})
$$hold. Note that $(A)_n = \bigcup_{i} (A_i)_n, (B)^n = \bigcup_{j} (B_j)^n$, one has\begin{align*}
2^{-dn}(\mathcal{E}^*(E, 2^{-n}) - \mathcal{E}_*(E, 2^{-n})) &\leq 2^{-dn}(\mathcal{E}_*((B)^n, 2^{-n}) - 
\mathcal{E}^*((A)_n, 2^{-n})) \leq m((B)^n) - m((A)_n)\\
&= m(B) - m(A) + m((B)^n\setminus B) + m(A \setminus (A)_n)\\
&\leq \varepsilon + m((B)^n\setminus B) + m(A \setminus (A)_n).
\end{align*}
Let $n \to \infty$, one has that $m((B)^n\setminus B) + m(A \setminus (A)_n)$ tends to zero. Thus$$
0 \leq 2^{-dn}(\mathcal{E}^*(E, 2^{-n}) - \mathcal{E}_*(E, 2^{-n})) \leq \varepsilon
$$holds for arbitary $\varepsilon > 0$. Let $\varepsilon$ tends to zero, we finally obtain that$$
\lim_{n \to \infty} 2^{-dn}(\mathcal{E}^*(E, 2^{-n}) - \mathcal{E}_*(E, 2^{-n})) = 0.
$$
\end{proof}

\begin{ex}[Uniqueness of Jordan  measure]\end{ex}
\begin{proof}
We use the conclusion in the elementary measure case so that $m'([0, r)^d) = cr^d, \forall r \in \mathbb{R}$ where 
$c = m'([0, 1)^d) \in \mathbb{R}^+$. Using previous notation, from finite additivity property one finds that 
$c2^{-dn}\mathcal{E}_*(E, 2^{-n}) \leq m'(E) \leq c2^{-dn}\mathcal{E}^*(E, 2^{-n})$. Since $E \in \mathcal{J}(\mathbb{R}^d)$,
let $n \to \infty$ and we finally obtain that $m'(E) = c2^{-dn}\mathcal{E}_*(E, 2^{-n}) = cm'(E)$ for all Jordan measurable
sets $E$.
\end{proof}

\begin{ex}\end{ex}
\begin{proof}
Since $E_1, E_2$ are Jordan measurable, there exists elementary sets $A_1, A_2, B_1, B_2$ s.t. 
$A_1 \subset E_1 \subset B_1, A_2 \subset E_2 \subset B_2$ which satisfy $m(B_i \setminus A_i) < \varepsilon, i = 1, 2$.
Note that $A_1 \times A_2 \subset E_1 \times E_2 \subset B_1 \times B_2$. We use the conclusion in the elementary
measure, we find that \begin{gather*}
m^{d_1 + d_2}(B_1 \times B_2) = m^{d_1}(B_1)m^{d_2}(B_2) \leq (m^{d_1}(A_1) + \varepsilon)(m^{d_2}(A_2) + \varepsilon) 
\leq (m^{d_1}(E_1) + \varepsilon)(m^{d_2}(E_2) + \varepsilon),\\
m^{d_1 + d_2}(A_1 \times A_2) = m^{d_1}(A_1)m^{d_2}(A_2) \geq (m^{d_1}(B_1) - \varepsilon)(m^{d_2}(B_2) - \varepsilon) 
\geq (m^{d_1}(E_1) - \varepsilon)(m^{d_2}(E_2) - \varepsilon).
\end{gather*}
From these two inequalities we obtain$$
m^{d_1 + d_2}(B_1 \times B_2) - m^{d_1 + d_2}(A_1 \times A_2) \leq 2(m^{d_1}(E_1) + m^{d_2}(E_2))\varepsilon.
$$Hence $E_1 \times E_2$ is Jordan measurable, and $m^{d_1 + d_2}(E_1 \times E_2) = m^{d_1}(E_1)\times m^{d_2}(E_2)$.
\end{proof}

\setcounter{ex}{17}
\begin{ex}\end{ex}
\begin{proof}
(1)Since $E \subset \overline{E}$, from the monotonicity we obtain $m^{*,(J)}(E) \leq m^{*,(J)}(\overline{E})$ 
immediately. For every $\varepsilon > 0$, there exists an elementary sets $B$ such that $B \supset E$ and inequality
$m(B) - m^{*, (J)}(E) < \varepsilon$ holds. Notice that $\overline{E} \subset \overline{B}$, so 
$m^{*, (J)}(\overline{E}) \leq m(B) \leq m^{*, (J)}(E) + \varepsilon$. Let $\varepsilon$ tends zero we obtain that
$m^{*, (J)}(E) \geq m^{*, (J)}(\overline{E})$. Hence $E$ and the closure $\overline{E}$ have the same Jordan outer
measure.\\
(2)Inequality $m_{*,(J)}(E^{\circ}) \leq m_{*,(J)}(E)$ is obvious for $E^{\circ} \subset E$. On the other hand, 
for every $\varepsilon > 0$, there exists an elementary set $A \subset E$ s.t. $m_{*,(J)}(E) - m(A) \leq \varepsilon$.
Note that $A^{\circ} \subset E^{\circ}$, thus $m_{*,(J)}(E^{\circ}) \geq m(A^{\circ}) = m(A) \geq m_{*,(J)}(E) - \varepsilon$,
since $\varepsilon$ is arbitary, $m_{*,(J)}(E^{\circ}) \geq m_{*,(J)}(E)$. Hence $E$ and the interior $E^\circ$ have
the same Jordan inner measure.\\
(3) $E$ is Jordan measurable if and only if $\lim_{n \to \infty} 2^{-dn}(\mathcal{E}^*(E, 2^{-n}) - \mathcal{E}_*(E, 2^{-n})) = 0$,
Notice that if $D$ is a dyadic cube that is not contained in $E$ but intersets $E$ if and only if it intersect $\partial E$,
that is $\mathcal{E}^*(\partial E, 2^{-n}) = \mathcal{E}^*(E, 2^{-n}) - \mathcal{E}_*(E, 2^{-n})$.\\
On the other hand, since $(\partial E)^\circ = \emptyset$, so $\partial E$ contains no dyadic cubes, i.e. 
$\lim_{n \to \infty} 2^{-dn}\mathcal{E}_*(\partial E, 2^{-n}) = 0 \Rightarrow m_{*, (J)} = 0$. And 
$\lim_{n \to \infty} 2^{-dn}\mathcal{E}^*(\partial E, 2^{-n}) = 0 = \lim_{n \to \infty} 2^{-dn}\mathcal{E}_*(\partial E, 2^{-n})$
if and only if $\partial E$ is Jordan measurable, if and only if $m^{*, (J)}(\partial E) = m_{*, (J)}(\partial E) = 0$.\\
(4)Since both of bullet-riddled square $[0, 1]^2 \setminus \mathbb{Q}^2$ and set of bullets $[0, 1]^2 \cap \mathbb{Q}^2$
have the interior $\emptyset$, so both have Jordan inner measure zero. And both have the closure $[0, 1]^2$, thus
they Jordan outer measure one. In particular, both set are not Jordan measurable.
\end{proof}

\begin{ex}[Carath\'eodory type property]\end{ex}
\begin{proof}
Let $E'$ be elementary sets s.t. $E' \supset E$, then $E' = (E' \cap F) \cup (E' \setminus F), E' \cap F 
\supset E \cap F, E' \setminus F \supset E \setminus F$. And the equality
$m(E') = m(E' \cap F) + m'(E' \setminus F)$ holds. We take infimum on both sides and find that $$
\inf_{E' \supset E} m(E') = \inf_{E' \supset E} (m(E' \cap F) + m(E' \setminus F)) \geq \inf_{A \supset E \cap F} m(A)
+ \inf_{B \supset E \setminus F} m(B),
$$where $A, B$ are elementary sets. Thus $m^{*, (J)}(E) \geq m^{*, (J)}(E \cap F) + m^{*, (J)}(E \setminus F)$.
Note that $E \subset(A \cup B) \setminus (A \cap B)$, and the equality $m((A \cup B) \setminus 
(A \cap B)) = m(A) + m(B) - m(A \cap B)$ holds. we firstly take infimum on $A$, then we take infimum on $B$, we obtain\begin{align*}
&m^{*,(J)}(E \cap F) + m^{*, (J)}(E \setminus F) - \inf_{B \supset E \setminus F} \inf_{A \supset E \cap F} m(A \cap B)\\
={} &\inf_{B \supset E \setminus F} \inf_{A \supset E \cap F} m((A \cup B) \setminus (A \cap B))
\geq \inf_{E' \supset E} m(E') = m^{*, (J)}(E).
\end{align*}
Since $m(A \cap B) \geq 0$, we obtain that $m^{*, (J)}(E) \leq m^{*, (J)}(E \cap F) + m^{*, (J)}(E \setminus F)$. 
From this inequality and the previous one we finally complete the proof$$
m^{*, (J)}(E) = m^{*, (J)}(E \cap F) + m^{*, (J)}(E \setminus F).
$$
\end{proof}
\end{document}